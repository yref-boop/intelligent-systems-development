\documentclass[12pt,a4paper,twoside,spanish]{article}      % Libro a 11 pt
\usepackage[utf8]{inputenc}
\usepackage[height=17.5cm,width=13.5cm]{geometry}
\usepackage[spanish]{babel}         % diccionario
\usepackage{epsfig}         % Graficos Postscript
\usepackage{tabularx}
\usepackage{sectsty}
\usepackage{float}

%color del texto
\usepackage {xcolor}


%%%%%%%%%%%%%%%%%%%%%%%%%%%%%%%%%%%%%%%%%%%%%%%
%%%%%%%%%%%%%
%%%%%%%%%%%%% Margenes
%%%%%%%%%%%%%
%%%%%%%%%%%%%%%%%%%%%%%%%%%%%%%%%%%%%%%%%%%%%%%
%%%%% Definimos el maximo tamaño posible.
\marginparwidth 0pt     \marginparsep 0pt
\topmargin   0pt        \textwidth   6.5in
\textheight 23cm

% Margen izq del txt en impares.
\setlength{\oddsidemargin}{.0001\textwidth}

% Margen izq del txt en pares.
\setlength{\evensidemargin}{-.04\textwidth}

% Anchura del texto
\setlength{\textwidth}{.99\textwidth}


%%%%%%%%%%%%%%%%%%%%%%%%%%%%%%%%%%%%%%%%%%%%%%%
%%%%%%%%%%%%%
%%%%%%%%%%%%% Profundidad de enumeracion y tabla de contenidos
%%%%%%%%%%%%%
%%%%%%%%%%%%%%%%%%%%%%%%%%%%%%%%%%%%%%%%%%%%%%%

\setcounter{secnumdepth}{3}
\setcounter{tocdepth}{3}


%%%%%%%%%%%%%%%%%%%%%%%%%%%%%%%%%%%%%%%%%%%%%%%
%%%%%%%%%%%%%
%%%%%%%%%%%%% Nuevos Comandos
%%%%%%%%%%%%%
%%%%%%%%%%%%%%%%%%%%%%%%%%%%%%%%%%%%%%%%%%%%%%%

            %%%%%%%%%%%%%%%%%%%%%%%
            %%%%%%%%%%%%%%%%%%%%%%%
            % Comandos para simplificar
            % la escritura
            %%%%%%%%%%%%%%%%%%%%%%%
            %%%%%%%%%%%%%%%%%%%%%%%

\def\mc{\multicolumn}
            %%%%%%%%%%%%%%%%%%%%%%%
            % Comandos para poder utilizar raggedright en tablas
            %%%%%%%%%%%%%%%%%%%%%%%
\newcommand{\PreserveBackslash}[1]{\let\temp=\\#1\let\\=\temp}
\let\PBS=\PreserveBackslash




%%%%%%%%%%%%%%%%%%%%%%%%%%%%%%%%%%%%%%%%%%%%%%%
%%%%%%%%%%%%%
%%%%%%%%%%%%% Cuerpo del documento
%%%%%%%%%%%%%
%%%%%%%%%%%%%%%%%%%%%%%%%%%%%%%%%%%%%%%%%%%%%%%


\begin{document}

\def\chaptername{Capítulo}
\def\tablename{Tabla}
\def\listtablename{Índice de Tablas}
\chapterfont{\LARGE\raggedleft}

%%%%%%%%%%%%%%%%%%%%%%%%%%%%%%%%%%%%%%%%%%%%%%%%%%%%%%%%%%%%%%%
%%%%%%%%%%%%%%%%%%%%%%%%%%%%%%%%%%%%%%%%%%%%%%%%%%%%%%%%%%%%%%%
% DISEÑO DE LA PAGINA DEL TITULO
%%%%%%%%%%%%%%%%%%%%%%%%%%%%%%%%%%%%%%%%%%%%%%%%%%%%%%%%%%%%%%%
%%%%%%%%%%%%%%%%%%%%%%%%%%%%%%%%%%%%%%%%%%%%%%%%%%%%%%%%%%%%%%%
\pagestyle{empty}

\begin{titlepage}
\setlength{\parindent}{0cm} \setlength{\parskip}{0cm}

%\raggedleft {\textsf{\textbf{Login del grupo: xxxx}}}

\newcommand{\HRule}{\rule{\linewidth}{1mm}}

\vspace*{2cm}
\HRule \\[0.5cm]
\begin{center}
% Letra lineal y negrita
\textsf{\textbf{\large UN SISTEMA BASADO EN CONOCIMIENTO PARA EL DISEÑO ÓPTIMO DE UN COCHE DE F1)\\[0.75cm] MODELADO CONCEPTUAL EN COMMONKADS \\[0.5cm]}}
\HRule \vspace*{3cm}

\textsf{\textbf{\normalsize Yago Fernández Rego\\ Rodrigo Naranjo González\\ Adrián Rodríguez López\\[5cm]
Horario de prácticas: Viernes 08:30\\
Login de entrega: r.naranjo\\[2.5cm]
Desarrollo de Sistemas Inteligentes\\
Universidade da Coruña \\ Curso
2023}}
\end{center}
\end{titlepage}

\cleardoublepage

%%%%%%%%%%%%%%%%%%%%%%%%%%%%%%%%%%%%%%%%%%%%%%%
%%
%% TABLA DE CONTENIDOS
%%
%%%%%%%%%%%%%%%%%%%%%%%%%%%%%%%%%%%%%%%%%%%%%%%

\pagenumbering{Roman}
\tableofcontents
\cleardoublepage


%%%%%%%%%%%%%%%%%%%%%%%%%%%%%%%%%%%%%%%%%%%%%%%%%%%%%%%%%%%%%%%
%%%%%%%%%%%%%%%%%%%%%%%%%%%%%%%%%%%%%%%%%%%%%%%%%%%%%%%%%%%%%%%
%CONTENIDO DEL DOCUMENTO
%%%%%%%%%%%%%%%%%%%%%%%%%%%%%%%%%%%%%%%%%%%%%%%%%%%%%%%%%%%%%%%
%%%%%%%%%%%%%%%%%%%%%%%%%%%%%%%%%%%%%%%%%%%%%%%%%%%%%%%%%%%%%%%

%numeros arábigos
\pagenumbering{arabic} \pagestyle{myheadings} \markboth{Grupo de
prácticas: xxx}{Modelado Conceptual en CommonKADS.}

%indentaciones y espaciado entre párrafos
\setlength{\parindent}{1,5cm} \setlength{\parskip}{0,7cm}


%%%%%%%%%%%%%%%%%%%%%%%%%%%%%%%%%%%%%%%%%%%%%%%%%%%%%%%%%%%%%%%
%% BORRAR
%%%%%%%%%%%%%%%%%%%%%%%%%%%%%%%%%%%%%%%%%%%%%%%%%%%%%%%%%%%%%%%
\textcolor {red} {Esta entrega se corresponde con los modelos de conocimiento y comunicación.  No debéis olvidar que la práctica supone el 50\% de la nota de la asignatura y que, sólo a nivel general, el material utilizado en las prácticas entrará en el examen teórico. Por esta razón esta segunda entrega supone un ``exame'' de los modelos de conocimiento y comunicación en lo que se refiere a los capítulos 5, 6, 7 y 9 del libro ``Knowledge engineering and management. The CommonKADS methodology"(Schreiber et col.)''. En las explicaciones que siguen me referiré siempre al libro de Schreiber et col. (aunque podéis encontrar el mismo material en el libro en español).}


%%%%%%%%%%%%%%%%%%%%%%%%%%%%%%%%%%%%%%%%%%%%%%%%%%%%%%%%%%%%%%%%%%%%%%%%%%%%%%%
\section {Modelo de Conocimiento.}
%%%%%%%%%%%%%%%%%%%%%%%%%%%%%%%%%%%%%%%%%%%%%%%%%%%%%%%%%%%%%%%%%%%%%%%%%%%%%%%

\textcolor {red} {Se deberá entregar la documentación correspondiente a las tres fases de diseño del modelo (explicadas en el capítulo 7 del libro de Shreiber).}

La tarea de la cual se refiere este modelo conceptual es la tarea 4 del OM-3, correspondiente a \textbf{Generación de la configuración optima del monoplaza}.
\begin{table}[H]
    \scriptsize
    \begin{tabularx}{\textwidth}{|p{0.2cm}|>{\raggedright}X|>{\raggedright}X|>{\raggedright}X|>{\raggedright}X|>{\raggedright}X|>{\PBS\raggedright}X|} \hline
        \multicolumn{3}{|l}{\textbf{Modelo de Organización}} &
        \multicolumn{4}{|l|}{\textbf{Formulario OM-3: Descomposición de los Procesos}}\\ \hline\hline \textsc{N\textordmasculine} &

    \textsc{Tarea} &
    \textsc{Realiza\-da por} &
    \textsc{¿Dónde?} &
    \textsc{Recursos de Conocimiento} &
    \textsc{¿In\-ten\-si\-va en Conocimiento?} &
    \textsc{Im\-por\-tan\-cia} \\ \hline

    4 &
    Generación de la configuración óptima del monoplaza. &
    Ingenieros y director del equipo. &
    \textit{Pit wall} del circuito donde se encuentran los monitores de la escudería. &
    Conocimiento del reglamento y experiencia en generación de configuraciones. &
    Sí, de razonamiento elevado. &
    Paso clave. \\ \hline

    \end{tabularx}
\end{table}

%%%%%%%%%%%%%%%%%%%%%%%%%%%%%%%%%%%%%%%%%%%%%%%%%%%%%%%%%%%%%%%%%%%%%%%%%%%%%%%
\subsection{Fase de Identificación}
%%%%%%%%%%%%%%%%%%%%%%%%%%%%%%%%%%%%%%%%%%%%%%%%%%%%%%%%%%%%%%%%%%%%%%%%%%%%%%%

\textcolor {red} {Indicación de la/s tarea/s del formulario OM-3 a las que se referirá el modelo de conocimiento.}

%%%%%%%%%%%%%%%%%%%%%%%%%%%%%%%%%%%%%%%%%%%%%%%%%%%%%%%%%%%%%%%%%%%%%%%%%%%%%%%
\subsubsection{Glosario}
%%%%%%%%%%%%%%%%%%%%%%%%%%%%%%%%%%%%%%%%%%%%%%%%%%%%%%%%%%%%%%%%%%%%%%%%%%%%%%%

\textcolor {red} {BREVE glosario de términos incluyendo sólo aquellos que no sean de uso común. Es decir, lo mínimo necesario para que una persona ajena al dominio que habéis elegido pueda entender el conocimiento que extraigáis.}

%%%%%%%%%%%%%%%%%%%%%%%%%%%%%%%%%%%%%%%%%%%%%%%%%%%%%%%%%%%%%%%%%%%%%%%%%%%%%%%
\subsubsection{Escenarios}
%%%%%%%%%%%%%%%%%%%%%%%%%%%%%%%%%%%%%%%%%%%%%%%%%%%%%%%%%%%%%%%%%%%%%%%%%%%%%%%

\textcolor {red} {Descripción de un par de escenarios de uso: ejemplos de entradas y salidas sobre los que se aplicará el sistema que proponéis. Tienen que ser compatibles son las entradas y salidas que habéis puesto en el TM-1 de la tarea que vais a implementar. Servirán para seleccionar la plantilla inferencial y para validar el modelo de conocimiento.}

%%%%%%%%%%%%%%%%%%%%%%%%%%%%%%%%%%%%%%%%%%%%%%%%%%%%%%%%%%%%%%%%%%%%%%%%%%%%%%%
\subsubsection{Elementos reutilizables}
%%%%%%%%%%%%%%%%%%%%%%%%%%%%%%%%%%%%%%%%%%%%%%%%%%%%%%%%%%%%%%%%%%%%%%%%%%%%%%%

\textcolor {red} {Elementos reutilizables que se hubiesen desarrollado con anterioridad, tales como esquemas de bases de datos ya existentes que podáis tomar como punto de partida para vuestro esquema del dominio. Probablemente no haya ninguno, pero...}


%%%%%%%%%%%%%%%%%%%%%%%%%%%%%%%%%%%%%%%%%%%%%%%%%%%%%%%%%%%%%%%%%%%%%%%%%%%%%%%
\subsection{Fase de Especificación}
%%%%%%%%%%%%%%%%%%%%%%%%%%%%%%%%%%%%%%%%%%%%%%%%%%%%%%%%%%%%%%%%%%%%%%%%%%%%%%%

\textcolor {red} {Los apartados de esta sección  se deberán de acompañar de BREVES comentarios indicando cuestiones sobre la adaptabilidad de la plantilla al problema, explicación de cambios que se hayan realizado sobre la misma, cómo se ha aplicado la metodología de construcción del modelo de conocimiento, etc.}

\textcolor {red} {EN DEFINITIVA, LOS COMENTARIOS DEBERÁN DEMOSTRAR QUE HABÉIS ENTENDIDO LOS PASOS EN LA CONSTRUCCION DEL MODELO Y, POR SUPUESTO, EL MODELO FINAL PROPUESTO.}

\textcolor {red} {Excepto para la tarea y su método, para la especificación de los demás componentes del modelo de conocimiento se utilizará la notación gráfica (frente a la textual de CML) siempre que sea posible. En el caso de los mapeos de inferencias podéis encontrar un ejemplo de notación gráfica en la página 200 del libro de Alonso et col. y en la página 107 en el libro de Schreiber et col.}


%%%%%%%%%%%%%%%%%%%%%%%%%%%%%%%%%%%%%%%%%%%%%%%%%%%%%%%%%%%%%%%%%%%%%%%%%%%%%%%
\subsubsection{Metodología empleada}
%%%%%%%%%%%%%%%%%%%%%%%%%%%%%%%%%%%%%%%%%%%%%%%%%%%%%%%%%%%%%%%%%%%%%%%%%%%%%%
\textcolor {red} {Justificación de la selección de la metodología \emph{Middle-in} o \emph{Middle-ou} a aplicar.}

%%%%%%%%%%%%%%%%%%%%%%%%%%%%%%%%%%%%%%%%%%%%%%%%%%%%%%%%%%%%%%%%%%%%%%%%%%%%%%%
\subsubsection{Plantilla anotada}
%%%%%%%%%%%%%%%%%%%%%%%%%%%%%%%%%%%%%%%%%%%%%%%%%%%%%%%%%%%%%%%%%%%%%%%%%%%%%%

\textcolor {red} {Gráfico con la plantilla notada. Todas las modificaciones propuestas sobre la plantilla original deberán justificarse en este punto.}

%%%%%%%%%%%%%%%%%%%%%%%%%%%%%%%%%%%%%%%%%%%%%%%%%%%%%%%%%%%%%%%%%%%%%%%%%%%%%%%
\subsubsection{Esquema inicial del dominio}
%%%%%%%%%%%%%%%%%%%%%%%%%%%%%%%%%%%%%%%%%%%%%%%%%%%%%%%%%%%%%%%%%%%%%%%%%%%%%%
\textcolor {red} {Boceto del que se parte que, posteriormente, se irá refinando al realizar el mapeo de las inferencias, etc.}

%%%%%%%%%%%%%%%%%%%%%%%%%%%%%%%%%%%%%%%%%%%%%%%%%%%%%%%%%%%%%%%%%%%%%%%%%%%%%%%
\subsubsection{Estructura inferencial y Mapeo}
%%%%%%%%%%%%%%%%%%%%%%%%%%%%%%%%%%%%%%%%%%%%%%%%%%%%%%%%%%%%%%%%%%%%%%%%%%%%%%

\textcolor {red} {Gráfico de la estruatura inferencial (solo si difiere de la plantilla seleccionada) \\ \\ Mapeo de cada una de las inferencias y roles de la plantilla seleccionada.}

%%%%%%%%%%%%%%%%%%%%%%%%%%%%%%%%%%%%%%%%%%%%%%%%%%%%%%%%%%%%%%%%%%%%%%%%%%%%%%%
\subsubsection{Tarea}
%%%%%%%%%%%%%%%%%%%%%%%%%%%%%%%%%%%%%%%%%%%%%%%%%%%%%%%%%%%%%%%%%%%%%%%%%%%%%%
\textcolor {red} {Especificación de la tarea (entradas y salidas) y Método de la tarea}

%%%%%%%%%%%%%%%%%%%%%%%%%%%%%%%%%%%%%%%%%%%%%%%%%%%%%%%%%%%%%%%%%%%%%%%%%%%%%%%
\subsubsection{Esquema del dominio final}
%%%%%%%%%%%%%%%%%%%%%%%%%%%%%%%%%%%%%%%%%%%%%%%%%%%%%%%%%%%%%%%%%%%%%%%%%%%%%%



%%%%%%%%%%%%%%%%%%%%%%%%%%%%%%%%%%%%%%%%%%%%%%%%%%%%%%%%%%%%%%%%%%%%%%%%%%%%%%%
\subsection{Fase De Refinamiento}
%%%%%%%%%%%%%%%%%%%%%%%%%%%%%%%%%%%%%%%%%%%%%%%%%%%%%%%%%%%%%%%%%%%%%%%%%%%%%%%


%%%%%%%%%%%%%%%%%%%%%%%%%%%%%%%%%%%%%%%%%%%%%%%%%%%%%%%%%%%%%%%%%%%%%%%%%%%%%%%
\subsubsection{Validación}
%%%%%%%%%%%%%%%%%%%%%%%%%%%%%%%%%%%%%%%%%%%%%%%%%%%%%%%%%%%%%%%%%%%%%%%%%%%%%%

\textcolor {red} {Validación del modelo sobre los escenarios propuestos en la fase de identificación.}

%%%%%%%%%%%%%%%%%%%%%%%%%%%%%%%%%%%%%%%%%%%%%%%%%%%%%%%%%%%%%%%%%%%%%%%%%%%%%%%
\subsubsection{Bases de conocimientos}
%%%%%%%%%%%%%%%%%%%%%%%%%%%%%%%%%%%%%%%%%%%%%%%%%%%%%%%%%%%%%%%%%%%%%%%%%%%%%%

\textcolor {red} {Para estas prácticas, bastará con que las bases de conocimientos contengan TRES o CUATRO instancias de cada tipo de regla encontrado.}

%%%%%%%%%%%%%%%%%%%%%%%%%%%%%%%%%%%%%%%%%%%%%%%%%%%%%%%%%%%%%%%%%%%%%%%%%%%%%%%
%%%%%%%%%%%%%%%%%%%%%%%%%%%%%%%%%%%%%%%%%%%%%%%%%%%%%%%%%%%%%%%%%%%%%%%%%%%%%%%
\section{Modelo de Comunicación}
%%%%%%%%%%%%%%%%%%%%%%%%%%%%%%%%%%%%%%%%%%%%%%%%%%%%%%%%%%%%%%%%%%%%%%%%%%%%%%%
%%%%%%%%%%%%%%%%%%%%%%%%%%%%%%%%%%%%%%%%%%%%%%%%%%%%%%%%%%%%%%%%%%%%%%%%%%%%%%%

\textcolor {red} {El modelado conceptual no está completo sin el modelo de comunicación, que es necesario realizar antes de la implementación. El modelo se realizará siguiendo las pautas de la metodología CommonKADS resumidas en la tabla 9.8 y se comprobará su corrección con respecto a los otros modelos ya construidos siguiendo las guías que se proporcionan en el apartado 9.7.4 del libro. En el documento a entregar se incluirán comentarios BREVES que demuestren que se han tenido en cuenta estos aspectos.}

%%%%%%%%%%%%%%%%%%%%%%%%%%%%%%%%%%%%%%%%%%%%%%%%%%%%%%%%%%%%%%%%%%%%%%%%%%%%%%%
\subsection{Plan de comunicación general}
%%%%%%%%%%%%%%%%%%%%%%%%%%%%%%%%%%%%%%%%%%%%%%%%%%%%%%%%%%%%%%%%%%%%%%%%%%%%%%

\textcolor {red}
{
Brevemente COMENTADO y que incluya:
\begin{itemize}
 \item Diagrama de diálogo
 \item Información de control de las transacciones en forma de
seudocódigo o de diagrama de transición
\end{itemize}
}

%%%%%%%%%%%%%%%%%%%%%%%%%%%%%%%%%%%%%%%%%%%%%%%%%%%%%%%%%%%%%%%%%%%%%%%%%%%%%%%
\subsection{Descripción de las Transacciones}
%%%%%%%%%%%%%%%%%%%%%%%%%%%%%%%%%%%%%%%%%%%%%%%%%%%%%%%%%%%%%%%%%%%%%%%%%%%%%%
\textcolor {red} {Modelo CM-1 que incluya las transacciones identificadas.}


%%%%%%%%%%%%%%%%%%%%%%%%%%%%%%%%%%%%%%%%%%%%%%%%%%%%%%%%%%%%%%%%%%%%%%%%%%%%%%%
\subsection{Especificación de las transacciones}
%%%%%%%%%%%%%%%%%%%%%%%%%%%%%%%%%%%%%%%%%%%%%%%%%%%%%%%%%%%%%%%%%%%%%%%%%%%%%%

\textcolor {red} {
Modelo CM-2 para las transacciones identificadas.
\\
\textbf{NOTA IMPORTANTE:} Normalmente los modelos de comunicación de las prácticas son sencillos, por lo que este formulario puede no ser necesario. Se realizará sólo
para las transacciones complejas que no queden completamente
especificadas con el modelo CM-1.
}

%%%%%%%%%%%%%%%%%%%%%%%%%%%%%%%%%%%%%%%%%%%%%%%%%%%%%%%%%%%%%%%
%FINAL DEL LIBRO
%%%%%%%%%%%%%%%%%%%%%%%%%%%%%%%%%%%%%%%%%%%%%%%%%%%%%%%%%%%%%%%
\end{document}
