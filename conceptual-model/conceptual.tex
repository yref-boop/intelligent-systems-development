\documentclass[12pt,a4paper,twoside,spanish]{article}      % Libro a 11 pt
\usepackage[utf8]{inputenc}
\usepackage[height=17.5cm,width=13.5cm]{geometry}
\usepackage[spanish]{babel}         % diccionario
\usepackage{epsfig}         % Graficos Postscript
\usepackage{tabularx}
\usepackage{sectsty}
\usepackage{float}

% color del texto
\usepackage {xcolor}

% diagramas
\usepackage{tikz}
\usetikzlibrary{shapes, arrows.meta, positioning}
\tikzset {font=\scriptsize}

% uml
\usepackage{pgf-umlcd}
\renewcommand{\umltextcolor}{black}
\renewcommand{\umldrawcolor}{black}
\renewcommand{\umlfillcolor}{white}

% arrows
\usetikzlibrary{arrows.meta}

% code
\usepackage{listings}
\lstset{escapeinside=||}
\lstset{
    numbers=left,
    numberstyle=\small,
    numbersep=8pt,
    frame = single,
    language=Pascal,
    framexleftmargin=20pt,
    framexrightmargin=20pt
}



%%%%%%%%%%%%%%%%%%%%%%%%%%%%%%%%%%%%%%%%%%%%%%%
%%%%%%%%%%%%%
%%%%%%%%%%%%% Margenes
%%%%%%%%%%%%%
%%%%%%%%%%%%%%%%%%%%%%%%%%%%%%%%%%%%%%%%%%%%%%%
%%%%% Definimos el maximo tamaño posible.
\marginparwidth 0pt     \marginparsep 0pt
\topmargin   0pt        \textwidth   6.5in
\textheight 23cm

% Margen izq del txt en impares.
\setlength{\oddsidemargin}{.0001\textwidth}

% Margen izq del txt en pares.
\setlength{\evensidemargin}{-.04\textwidth}

% Anchura del texto
\setlength{\textwidth}{.99\textwidth}


%%%%%%%%%%%%%%%%%%%%%%%%%%%%%%%%%%%%%%%%%%%%%%%
%%%%%%%%%%%%% %%%%%%%%%%%%% Profundidad de enumeracion y tabla de contenidos
%%%%%%%%%%%%%
%%%%%%%%%%%%%%%%%%%%%%%%%%%%%%%%%%%%%%%%%%%%%%%

\setcounter{secnumdepth}{3}
\setcounter{tocdepth}{3}


%%%%%%%%%%%%%%%%%%%%%%%%%%%%%%%%%%%%%%%%%%%%%%%
%%%%%%%%%%%%%
%%%%%%%%%%%%% Nuevos Comandos
%%%%%%%%%%%%%
%%%%%%%%%%%%%%%%%%%%%%%%%%%%%%%%%%%%%%%%%%%%%%%

            %%%%%%%%%%%%%%%%%%%%%%%
            %%%%%%%%%%%%%%%%%%%%%%%
            % Comandos para simplificar
            % la escritura
            %%%%%%%%%%%%%%%%%%%%%%%
            %%%%%%%%%%%%%%%%%%%%%%%

\def\mc{\multicolumn}
            %%%%%%%%%%%%%%%%%%%%%%%
            % Comandos para poder utilizar raggedright en tablas
            %%%%%%%%%%%%%%%%%%%%%%%
\newcommand{\PreserveBackslash}[1]{\let\temp=\\#1\let\\=\temp}
\let\PBS=\PreserveBackslash

\newcommand {\inference}[5] {
    \begin {tikzpicture}
         \node [draw,
            minimum width=2cm,
            minimum height=1cm] at (0,0) (input) {#1};
        \node [draw,
            rounded corners = 0.5cm,
            minimum width=2cm,
            minimum height=1cm] at (3,0) (inference) {#2};
        \node [draw,
            minimum width=2cm,
            minimum height=1cm] at (6,0) (output) {#3};
        \node [draw,
            minimum width=2cm,
            minimum height=1cm] at (0,-2) (input_type) {#4};
        \node [draw,
            minimum width=2cm,
            minimum height=1cm] at (6,-2) (output_type) {#5};

        \draw [-latex] (input) to (inference);
        \draw [-latex] (inference) to (output);
        \draw [dotted] (-1,-1) to (7,-1);
        \draw [->,dashed] (input) edge (input_type);
        \draw [->,dashed] (output) edge (output_type);
    \end {tikzpicture}
}


%%%%%%%%%%%%%%%%%%%%%%%%%%%%%%%%%%%%%%%%%%%%%%%
%%%%%%%%%%%%%
%%%%%%%%%%%%% Cuerpo del documento
%%%%%%%%%%%%%
%%%%%%%%%%%%%%%%%%%%%%%%%%%%%%%%%%%%%%%%%%%%%%%


\begin{document}

\def\chaptername{Capítulo}
\def\tablename{Tabla}
\def\listtablename{Índice de Tablas}
\chapterfont{\LARGE\raggedleft}

%%%%%%%%%%%%%%%%%%%%%%%%%%%%%%%%%%%%%%%%%%%%%%%%%%%%%%%%%%%%%%%
%%%%%%%%%%%%%%%%%%%%%%%%%%%%%%%%%%%%%%%%%%%%%%%%%%%%%%%%%%%%%%%
% DISEÑO DE LA PAGINA DEL TITULO
%%%%%%%%%%%%%%%%%%%%%%%%%%%%%%%%%%%%%%%%%%%%%%%%%%%%%%%%%%%%%%%
%%%%%%%%%%%%%%%%%%%%%%%%%%%%%%%%%%%%%%%%%%%%%%%%%%%%%%%%%%%%%%%
\pagestyle{empty}

\begin{titlepage}
\setlength{\parindent}{0cm} \setlength{\parskip}{0cm}

%\raggedleft {\textsf{\textbf{Login del grupo: xxxx}}}

\newcommand{\HRule}{\rule{\linewidth}{1mm}}

\vspace*{2cm}
\HRule \\[0.5cm]
\begin{center}
% Letra lineal y negrita
\textsf{\textbf{\large UN SISTEMA BASADO EN CONOCIMIENTO PARA EL DISEÑO ÓPTIMO DE UN COCHE DE F1)\\[0.75cm] MODELADO CONCEPTUAL EN COMMONKADS \\[0.5cm]}}
\HRule \vspace*{3cm}

\textsf{\textbf{\normalsize Yago Fernández Rego\\ Rodrigo Naranjo González\\ Adrián Rodríguez López\\[5cm]
Horario de prácticas: Viernes 08:30\\
Login de entrega: r.naranjo\\[2.5cm]
Desarrollo de Sistemas Inteligentes\\
Universidade da Coruña \\ Curso
2023}}
\end{center}
\end{titlepage}

\cleardoublepage

%%%%%%%%%%%%%%%%%%%%%%%%%%%%%%%%%%%%%%%%%%%%%%%
%%
%% TABLA DE CONTENIDOS
%%
%%%%%%%%%%%%%%%%%%%%%%%%%%%%%%%%%%%%%%%%%%%%%%%

\pagenumbering{Roman}
\tableofcontents
\cleardoublepage


%%%%%%%%%%%%%%%%%%%%%%%%%%%%%%%%%%%%%%%%%%%%%%%%%%%%%%%%%%%%%%%
%%%%%%%%%%%%%%%%%%%%%%%%%%%%%%%%%%%%%%%%%%%%%%%%%%%%%%%%%%%%%%%
%CONTENIDO DEL DOCUMENTO
%%%%%%%%%%%%%%%%%%%%%%%%%%%%%%%%%%%%%%%%%%%%%%%%%%%%%%%%%%%%%%%
%%%%%%%%%%%%%%%%%%%%%%%%%%%%%%%%%%%%%%%%%%%%%%%%%%%%%%%%%%%%%%%

%numeros arábigos
\pagenumbering{arabic} \pagestyle{myheadings} \markboth{Grupo de prácticas: 3.1}{Modelado Conceptual en CommonKADS.}

%indentaciones y espaciado entre párrafos
\setlength{\parindent}{1,5cm} \setlength{\parskip}{0,7cm}


%%%%%%%%%%%%%%%%%%%%%%%%%%%%%%%%%%%%%%%%%%%%%%%%%%%%%%%%%%%%%%%
%% BORRAR
%%%%%%%%%%%%%%%%%%%%%%%%%%%%%%%%%%%%%%%%%%%%%%%%%%%%%%%%%%%%%%%
\textcolor {red} {Esta entrega se corresponde con los modelos de conocimiento y comunicación.  No debéis olvidar que la práctica supone el 50\% de la nota de la asignatura y que, sólo a nivel general, el material utilizado en las prácticas entrará en el examen teórico. Por esta razón esta segunda entrega supone un ``exame'' de los modelos de conocimiento y comunicación en lo que se refiere a los capítulos 5, 6, 7 y 9 del libro ``Knowledge engineering and management. The CommonKADS methodology"(Schreiber et col.)''. En las explicaciones que siguen me referiré siempre al libro de Schreiber et col. (aunque podéis encontrar el mismo material en el libro en español).}


%%%%%%%%%%%%%%%%%%%%%%%%%%%%%%%%%%%%%%%%%%%%%%%%%%%%%%%%%%%%%%%%%%%%%%%%%%%%%%%
\section {Modelo de Conocimiento.}
%%%%%%%%%%%%%%%%%%%%%%%%%%%%%%%%%%%%%%%%%%%%%%%%%%%%%%%%%%%%%%%%%%%%%%%%%%%%%%%

\textcolor {red} {Se deberá entregar la documentación correspondiente a las tres fases de diseño del modelo (explicadas en el capítulo 7 del libro de Shreiber).}


%%%%%%%%%%%%%%%%%%%%%%%%%%%%%%%%%%%%%%%%%%%%%%%%%%%%%%%%%%%%%%%%%%%%%%%%%%%%%%%
\subsection{Fase de Identificación}
%%%%%%%%%%%%%%%%%%%%%%%%%%%%%%%%%%%%%%%%%%%%%%%%%%%%%%%%%%%%%%%%%%%%%%%%%%%%%%%

La tarea de la cual se refiere este modelo conceptual es la tarea 4 del OM-3, correspondiente a \textbf{Generación de la configuración optima del monoplaza}.
\begin{table}[H]
    \scriptsize
    \begin{tabularx}{\textwidth}{|p{0.2cm}|>{\raggedright}X|>{\raggedright}X|>{\raggedright}X|>{\raggedright}X|>{\raggedright}X|>{\PBS\raggedright}X|} \hline
        \multicolumn{3}{|l}{\textbf{Modelo de Organización}} &
        \multicolumn{4}{|l|}{\textbf{Formulario OM-3: Descomposición de los Procesos}}\\ \hline\hline \textsc{N\textordmasculine} &

    \textsc{Tarea} &
    \textsc{Realiza\-da por} &
    \textsc{¿Dónde?} &
    \textsc{Recursos de Conocimiento} &
    \textsc{¿In\-ten\-si\-va en Conocimiento?} &
    \textsc{Im\-por\-tan\-cia} \\ \hline

    4 &
    Generación de la configuración óptima del monoplaza. &
    Ingenieros y director del equipo. &
    \textit{Pit wall} del circuito donde se encuentran los monitores de la escudería. &
    Conocimiento del reglamento y experiencia en generación de configuraciones. &
    Sí, de razonamiento elevado. &
    Paso clave. \\ \hline

    \end{tabularx}
\end{table}

%%%%%%%%%%%%%%%%%%%%%%%%%%%%%%%%%%%%%%%%%%%%%%%%%%%%%%%%%%%%%%%%%%%%%%%%%%%%%%%
\subsubsection{Glosario}
%%%%%%%%%%%%%%%%%%%%%%%%%%%%%%%%%%%%%%%%%%%%%%%%%%%%%%%%%%%%%%%%%%%%%%%%%%%%%%%

\textcolor {red} {BREVE glosario de términos incluyendo sólo aquellos que no sean de uso común. Es decir, lo mínimo necesario para que una persona ajena al dominio que habéis elegido pueda entender el conocimiento que extraigáis.}

%%%%%%%%%%%%%%%%%%%%%%%%%%%%%%%%%%%%%%%%%%%%%%%%%%%%%%%%%%%%%%%%%%%%%%%%%%%%%%%
\subsubsection{Escenarios}
%%%%%%%%%%%%%%%%%%%%%%%%%%%%%%%%%%%%%%%%%%%%%%%%%%%%%%%%%%%%%%%%%%%%%%%%%%%%%%%

\textcolor {red} {Descripción de un par de escenarios de uso: ejemplos de entradas y salidas sobre los que se aplicará el sistema que proponéis. Tienen que ser compatibles son las entradas y salidas que habéis puesto en el TM-1 de la tarea que vais a implementar. Servirán para seleccionar la plantilla inferencial y para validar el modelo de conocimiento.}

%%%%%%%%%%%%%%%%%%%%%%%%%%%%%%%%%%%%%%%%%%%%%%%%%%%%%%%%%%%%%%%%%%%%%%%%%%%%%%%
\subsubsection{Elementos reutilizables}
%%%%%%%%%%%%%%%%%%%%%%%%%%%%%%%%%%%%%%%%%%%%%%%%%%%%%%%%%%%%%%%%%%%%%%%%%%%%%%%

\textcolor {red} {Elementos reutilizables que se hubiesen desarrollado con anterioridad, tales como esquemas de bases de datos ya existentes que podáis tomar como punto de partida para vuestro esquema del dominio. Probablemente no haya ninguno, pero...}


%%%%%%%%%%%%%%%%%%%%%%%%%%%%%%%%%%%%%%%%%%%%%%%%%%%%%%%%%%%%%%%%%%%%%%%%%%%%%%%
\subsection{Fase de Especificación}
%%%%%%%%%%%%%%%%%%%%%%%%%%%%%%%%%%%%%%%%%%%%%%%%%%%%%%%%%%%%%%%%%%%%%%%%%%%%%%%

\textcolor {red} {Los apartados de esta sección  se deberán de acompañar de BREVES comentarios indicando cuestiones sobre la adaptabilidad de la plantilla al problema, explicación de cambios que se hayan realizado sobre la misma, cómo se ha aplicado la metodología de construcción del modelo de conocimiento, etc.}

\textcolor {red} {EN DEFINITIVA, LOS COMENTARIOS DEBERÁN DEMOSTRAR QUE HABÉIS ENTENDIDO LOS PASOS EN LA CONSTRUCCION DEL MODELO Y, POR SUPUESTO, EL MODELO FINAL PROPUESTO.}

\textcolor {red} {Excepto para la tarea y su método, para la especificación de los demás componentes del modelo de conocimiento se utilizará la notación gráfica (frente a la textual de CML) siempre que sea posible. En el caso de los mapeos de inferencias podéis encontrar un ejemplo de notación gráfica en la página 200 del libro de Alonso et col. y en la página 107 en el libro de Schreiber et col.}


%%%%%%%%%%%%%%%%%%%%%%%%%%%%%%%%%%%%%%%%%%%%%%%%%%%%%%%%%%%%%%%%%%%%%%%%%%%%%%%
\subsubsection{Metodología empleada}
%%%%%%%%%%%%%%%%%%%%%%%%%%%%%%%%%%%%%%%%%%%%%%%%%%%%%%%%%%%%%%%%%%%%%%%%%%%%%%
La metodología escogida para este proyecto es \textit{Middle-out} ya que la plantilla escogida funciona como aproximación correcta de la estructura de inferencias de manera bastante directa.


%%%%%%%%%%%%%%%%%%%%%%%%%%%%%%%%%%%%%%%%%%%%%%%%%%%%%%%%%%%%%%%%%%%%%%%%%%%%%%%
\subsubsection{Plantilla anotada}
%%%%%%%%%%%%%%%%%%%%%%%%%%%%%%%%%%%%%%%%%%%%%%%%%%%%%%%%%%%%%%%%%%%%%%%%%%%%%%

\textcolor {red} {Gráfico con la plantilla notada. Todas las modificaciones propuestas sobre la plantilla original deberán justificarse en este punto.}
La plantilla seleccionada es la de \textit{configuration design}, modificada ligeramente para adaptarse mejor a la naturaleza de nuestro problema:

\begin {tikzpicture}
    \node [draw,
        minimum width=2cm,
        minimum height=1cm] at (3,0) (skeletal design) {skeletal design};
    \node [draw,
        minimum width=2cm,
        minimum height=1cm] at (0,-2) (soft) {soft requirements};
    \node [draw,
        rounded corners = 0.5cm,
        minimum width=2cm,
        minimum height=1cm] (propose) at (3,-2) {propose};
    \node [draw,
        minimum width=2cm,
        minimum height=1cm] at (6,-2) (extension) {extension};
    \node [draw,
        minimum width=2cm,
        minimum height=1cm] at (9,-4) (hard) {hard requirements};
    \node [draw,
        minimum width=2cm,
        minimum height=1cm] at (3,-6) (design) {design};
    \node [draw,
        rounded corners = 0.5cm,
        minimum width=2cm,
        minimum height=1cm] at (6,-6) (verify) {verify};
    \node [draw,
        rounded corners = 0.5cm,
        minimum width=2cm,
        minimum height=1cm] at (0,-8) (modify) {modify};
    \node [draw,
        minimum width=2cm,
        minimum height=1cm] at (0,-10) (action) {action};
    \node [draw,
        rounded corners = 0.5cm,
        minimum width=2cm,
        minimum height=1cm] at (3,-10) (critique) {critique};
    \node [draw,
        minimum width=2cm,
        minimum height=1cm] at (6,-10) (violation) {violation};
    \node [draw,
        minimum width=2cm,
        minimum height=1cm] at (9,-10) (truth) {truth value};
    \node [draw,
        rounded corners = 0.5cm,
        minimum width=2cm,
        minimum height=1cm] at (0,-12) (select) {select};
    \node [draw,
        minimum width=2cm,
        minimum height=1cm] at (3,-12) (list) {action list};

    \draw [line width=0mm, black] (7.75,-5.5) -- (10.25,-5.5);
    \node [align=center,
        minimum width=2cm,
        minimum height=1cm] at (9,-6) (k_verify) {conocimiento de\\verificación};
    \draw [line width=0mm, black] (7.75,-6.5) -- (10.25,-6.5);
    \draw [-latex] (7.75,-6) to (verify);

    \draw [line width=0mm, black] (4.75,-11.5) -- (7.25,-11.5);
    \node [align=center,
        minimum width=2cm,
        minimum height=1cm] at (6,-12) (k_critique) {conocimiento de\\crítica};
    \draw [line width=0mm, black] (4.75,-12.5) -- (7.25,-12.5);
    \draw [-latex] (k_critique) to (critique);

    \draw [line width=0mm, black] (-1.25,-3.5) -- (1.25,-3.5);
    \node [align=center,
        minimum width=2cm,
        minimum height=1cm] at (0,-4) (k_propose) {conocimiento de\\propuesta};
    \draw [-latex] (k_propose) to (propose);
    \draw [line width=0mm, black] (-1.25,-4.5) -- (1.25,-4.5);

    \draw [-latex] (skeletal design) to (propose);
    \draw [-latex] (soft) to (propose);
    \draw [-latex] (propose) to (extension);
    \draw [-latex] (extension) to (verify);
    \draw [tips = on proper draw] (hard) to (6,-4);
    \draw [-latex] (verify) to (violation);
    \draw (modify) -- (0,-6);
    \draw [-latex] (0,-6) to (design);
    \draw [-latex] (design) to (critique);
    \draw [-latex] (3,-8) to (modify);
    \draw (6,-8) -- (9,-8);
    \draw [-latex] (9,-8) to (truth);
    \draw [-latex] (violation) to (critique);
    \draw [-latex] (critique) to (list);
    \draw [-latex] (list) to (select);
    \draw [-latex] (select) to (action);
    \draw [-latex] (action) to (modify);
    \draw [-latex] (design) to (propose);
    \draw [-latex] (design) to (verify);
\end {tikzpicture}

\subsubsection {Anotaciones}
\begin {description}
    \item [Cambios:] Debido a la naturaleza de los datos de entrada, consideramos innecesarias las inferencias \textit {specify} y \textit {operationalize}. Nuestros datos ya vienen formateados y separados de manera que se puedan introducir directamente en el sistema como \textit {soft} y \textit {hard requirements}, pasando estos dos a ser los inputs de nuestro sistema.
    \item [Soft requirements:] Requisitos a usar como preferencias. En nuestro caso, las preferencias de conducción del piloto.
    \item [Hard requirements:] Requisitos obligatorios. Ej: Necesario que el coche sea capaz de adaptarse a altas temperaturas si las temperaturas previstas son altas.
    \item [Skeletal design:] Colección de componentes que debe contener la solución: En nuestro caso, 4 ruedas, un motor, alerón,...
    \item [Extension:] Crear una extensión de diseño, basándonos en las preferencias de la base de conocimiento. Ej: Neumáticos duros.
    \item [Truth value:] Resultado booleano de verificar si el diseño propuesto es consistente o no. Ej: El coche una mayor cantidad de combustible al marcado en el reglamento por lo que devuelve false.
    \item [Violation:] Restricción incumplida por el diseño actual. Ej: Reglamento incumplido - Capacidad de combustible sobrepasada
    \item [Action list:] Lista ordenada de posibles acciones para arreglar problemas del diseño actual. Ej: Para "Reglamento incumplido - Capacidad de combustible sobrepasada", elegimos otro depósito con menor capacidad; para "Reglamento incumplido - Potencia del motor sobrepasada", elegimos otro con menor potencia.
    \item [Action:] Una simple acción para arreglar problemas del diseño actual. Ej: Para "Reglamento incumplido - Capacidad de combustible sobrepasada", elegimos otro depósito con menor capacidad.
    \item [Design:] Nuevo diseño propuesto tras aplicar el fix al diseño anterior. Ej: Otro coche con características similares con un depósito de combustible reglamentario.
\end {description}


%%%%%%%%%%%%%%%%%%%%%%%%%%%%%%%%%%%%%%%%%%%%%%%%%%%%%%%%%%%%%%%%%%%%%%%%%%%%%%%
\subsubsection{Esquema inicial del dominio}
%%%%%%%%%%%%%%%%%%%%%%%%%%%%%%%%%%%%%%%%%%%%%%%%%%%%%%%%%%%%%%%%%%%%%%%%%%%%%%

\begin{tikzpicture}
    \begin{class}[rectangle split parts=2, text width=4cm]{Requisitos}{0,0}
        \attribute{PreferenciasPiloto : piloto}
        \attribute{Circuito : circuito}
        \attribute{Meteorologia : meteorologia}
    \end{class}
    \begin{class}[rectangle split parts=2, yshift=-5mm, text width = 3.5cm]{Meteorologia}{Requisitos.south}
        \attribute{Float: mmAgua}
        \attribute{Float: temperatura}
        \attribute{Float: velocidadViento}
        \attribute{Float: presionAtmosferica}
        \inherit {Requisitos}
    \end{class}
    \begin{class}[rectangle split parts=2, xshift=-32mm, text width = 4cm]{PreferenciasPiloto}{Meteorologia.north west}
        \attribute {String : tipoConduccion}
        \inherit {Requisitos}
    \end{class}
    \begin{class}[rectangle split parts=2, xshift=+32mm, text width = 3cm]{Circuito}{Meteorologia.north east}
        \attribute{Float: pendiente}
        \attribute{Float: curvas}
        \attribute{Float: longitud}
        \inherit {Requisitos}
    \end{class}
    \begin {class} [rectangle split parts=2, yshift=-10mm, text width = 3cm] {Coche} {PreferenciasPiloto.south}
        \attribute {String : tipoRuedas}
        \attribute {String : motor}
        \attribute {String : aleron}
    \end {class}
\end{tikzpicture}

%%%%%%%%%%%%%%%%%%%%%%%%%%%%%%%%%%%%%%%%%%%%%%%%%%%%%%%%%%%%%%%%%%%%%%%%%%%%%%%
\subsubsection{Estructura inferencial y Mapeo}
%%%%%%%%%%%%%%%%%%%%%%%%%%%%%%%%%%%%%%%%%%%%%%%%%%%%%%%%%%%%%%%%%%%%%%%%%%%%%%

\textbf {Propose}
\\[0.5cm]
\inference {skeletal design}{propose}{extension}{design}{design element}
\hspace {1cm}
\inference {design}{propose}{extension}{design}{design element}
\\[1cm]
\inference {soft requirements}{propose}{extension}{preferences}{design element}
\\[1cm]
\textbf {Verify}
\\[0.5cm]
\inference {design}{verify}{truth value}{design}{boolean}
\hspace {1cm}
\inference {design}{verify}{violation}{design}{constraint}
\\[1cm]
\inference {hard requirements}{verify}{truth value}{properties}{boolean}
\hspace {1cm}
\inference {hard requirements}{verify}{violation}{properties}{constraint}
\\[1cm]
\inference {extension}{verify}{truth value}{design element}{boolean}
\hspace {1cm}
\inference {extension}{verify}{violation}{design element}{constraint}
\\[1cm]
\textbf {Critique}
\\[0.5cm]
\inference {violation}{critique}{action list}{constraint}{fixes}
\hspace {1cm}
\inference {design}{critique}{action list}{design}{fixes}
\\[1cm]
\textbf {Select}
\\[0.5cm]
\inference {action list}{select}{action}{fixes}{fixAction}
\\[1cm]
\textbf {Modify}
\\[0.5cm]
\inference {action}{modify}{design}{fixAction}{design}
\hspace {1cm}
\inference {design}{modify}{design}{fixAction}{design}

\textbf{Descripciones}
\begin{description}
    \item [Propose:] Crear un nuevo elemento a añadir al diseño
        \hspace*{2cm}
        \begin{itemize}
            \item \textbf {Ejemplo:} escoger la altura que va a tener el monoplaza
            \item \textbf {Conocimiento Estático:} dependencias entre elecciones de componentes; preferencias de componentes
            \item \textbf {Apuntes:} solo se genera parte de la solución.
        \end{itemize}
    \item [Critique:] Dada una propuesta de solución, genera uno o varios problemas relacionados.
       \begin{itemize}
            \item \textbf {Ejemplo:} Criticar un posible diseño del coche.
            \item \textbf {Conocimiento Estático:} Tendencia a ser heurísitca y dependiente del contexto.
            \item \textbf {Computación:} Actualización básica.
        \end{itemize}
    \item [Select:] Escoge un elemento de la lista o conjunto dado.
        \begin{itemize}
            \item \textbf {Ejemplo:} Específicamente escoge una acción a tomar, por ejemplo, reducir altura del monoplaza.
            \item \textbf {Conocimiento Estático:} Conocimiento del dominio.
            \item \textbf {Computación:} Forward reasoning.
        \end{itemize}
    \item [Modify:] Toma descripción del sistema tanto como input como output, con input opcional la acción concreta.
        \begin{itemize}
            \item \textbf {Ejemplo:} Modificar el diseño del coche cambiando la altura del modelo.
            \item \textbf {Conocimiento Estático:} Conocimiento sobre la acción; acción simple o repetible.
            \item \textbf {Computación:} Actualización básica.
        \end{itemize}
    \item [Verify:] Dado un diseño, emite un valor de verdad, indicando si el diseño pasa el test, en caso contrario indica la violación.
        \begin{itemize}
            \item \textbf {Ejemplo:} Verificar un diseño de monoplaza.
            \item \textbf {Conocimiento Estático:} Conocimiento del dominio.
            \item \textbf {Computación:}  Algoritmos de selección básicos.
            \item \textbf {Apuntes:} Es una buena candidata al uso de refinamiento gradual: presuponer selección aleatoria e ir añadiendo conocimiento sobre selección más adelante para optimizar el razonamiento.
        \end{itemize}
\end{description}

\newpage
%%%%%%%%%%%%%%%%%%%%%%%%%%%%%%%%%%%%%%%%%%%%%%%%%%%%%%%%%%%%%%%%%%%%%%%%%%%%%%%
\subsubsection{Tarea}
%%%%%%%%%%%%%%%%%%%%%%%%%%%%%%%%%%%%%%%%%%%%%%%%%%%%%%%%%%%%%%%%%%%%%%%%%%%%%%
\textbf {Especificación de la tarea}
\begin{lstlisting}
    TASK : configuration-design;
    INPUT :
      soft requirements : "preferencias de conducci|ó|n del piloto";
      hard requirements : "datos sobre el circuito y meteorolog|í|a";
    OUTPUT : dise|ñ|o : "propuesta de dise|ñ|o de un monoplaza";
    END TASK;
\end{lstlisting}

\textbf {Método de la tarea}
\begin{lstlisting}
    TASK-METHOD: propose-and-revise;
      REALIZES: configuration-design;
      DECOMPOSITION:
        INFERENCES: propose, verify, critique, select, modify;
      ROLES:
        INTERMEDIATE:
          skeletal-design: "conjunto de elementos de dise|ñ|o";
          extension: "un valor nuevo para un |ú|nico design element";
          violation: "constraint incumplido por el design actual";
          truth-value: "boolean valor de la verificaci|ó|n";
          action-list: "lista de posibles fixAction";
          action: "una |ú|nica acci|ó|n de reparaci|ó|n;
      CONTROL-STRUCTURE;
          WHILE NEW-SOLUTION propose(skeletal-design + design +
              soft-requirements -> extension) DO
            design := extension ADD design;
            verify(design + hard-requirements -> truth-value +
                violation);
            IF truth-value == false
            THEN
              critique(violation + design -> action-list);
              REPEAT
                select(action-list -> action);
                modify(design + action -> design);
                verify(design + hard-requirements -> truth-value +
                    violation);
              UNTIL truth-value == true;
              END REPEAT
            END IF
          END WHILE
    END TASK-METHOD propose-and-revise;
\end{lstlisting}

%%%%%%%%%%%%%%%%%%%%%%%%%%%%%%%%%%%%%%%%%%%%%%%%%%%%%%%%%%%%%%%%%%%%%%%%%%%%%%%
\subsubsection{Esquema del dominio final}
%%%%%%%%%%%%%%%%%%%%%%%%%%%%%%%%%%%%%%%%%%%%%%%%%%%%%%%%%%%%%%%%%%%%%%%%%%%%%%

%%%%%%%%%%%%%%%%%%%%%%%%%%%%%%%%%%%%%%%%%%%%%%%%%%%%%%%%%%%%%%%%%%%%%%%%%%%%%%%
\subsection{Fase De Refinamiento}
%%%%%%%%%%%%%%%%%%%%%%%%%%%%%%%%%%%%%%%%%%%%%%%%%%%%%%%%%%%%%%%%%%%%%%%%%%%%%%%


%%%%%%%%%%%%%%%%%%%%%%%%%%%%%%%%%%%%%%%%%%%%%%%%%%%%%%%%%%%%%%%%%%%%%%%%%%%%%%%
\subsubsection{Validación}
%%%%%%%%%%%%%%%%%%%%%%%%%%%%%%%%%%%%%%%%%%%%%%%%%%%%%%%%%%%%%%%%%%%%%%%%%%%%%%

\textcolor {red} {Validación del modelo sobre los escenarios propuestos en la fase de identificación.}

%%%%%%%%%%%%%%%%%%%%%%%%%%%%%%%%%%%%%%%%%%%%%%%%%%%%%%%%%%%%%%%%%%%%%%%%%%%%%%%
\subsubsection{Bases de conocimientos}
%%%%%%%%%%%%%%%%%%%%%%%%%%%%%%%%%%%%%%%%%%%%%%%%%%%%%%%%%%%%%%%%%%%%%%%%%%%%%%

\textcolor {red} {Para estas prácticas, bastará con que las bases de conocimientos contengan TRES o CUATRO instancias de cada tipo de regla encontrado.}

%%%%%%%%%%%%%%%%%%%%%%%%%%%%%%%%%%%%%%%%%%%%%%%%%%%%%%%%%%%%%%%%%%%%%%%%%%%%%%%
%%%%%%%%%%%%%%%%%%%%%%%%%%%%%%%%%%%%%%%%%%%%%%%%%%%%%%%%%%%%%%%%%%%%%%%%%%%%%%%
\section{Modelo de Comunicación}
%%%%%%%%%%%%%%%%%%%%%%%%%%%%%%%%%%%%%%%%%%%%%%%%%%%%%%%%%%%%%%%%%%%%%%%%%%%%%%%
%%%%%%%%%%%%%%%%%%%%%%%%%%%%%%%%%%%%%%%%%%%%%%%%%%%%%%%%%%%%%%%%%%%%%%%%%%%%%%%

\textcolor {red} {El modelado conceptual no está completo sin el modelo de comunicación, que es necesario realizar antes de la implementación. El modelo se realizará siguiendo las pautas de la metodología CommonKADS resumidas en la tabla 9.8 y se comprobará su corrección con respecto a los otros modelos ya construidos siguiendo las guías que se proporcionan en el apartado 9.7.4 del libro. En el documento a entregar se incluirán comentarios BREVES que demuestren que se han tenido en cuenta estos aspectos.}

%%%%%%%%%%%%%%%%%%%%%%%%%%%%%%%%%%%%%%%%%%%%%%%%%%%%%%%%%%%%%%%%%%%%%%%%%%%%%%%
\subsection{Plan de comunicación general}
%%%%%%%%%%%%%%%%%%%%%%%%%%%%%%%%%%%%%%%%%%%%%%%%%%%%%%%%%%%%%%%%%%%%%%%%%%%%%%

\textcolor {red}
{
Brevemente COMENTADO y que incluya:
\begin{itemize}
 \item Diagrama de diálogo
 \item Información de control de las transacciones en forma de
seudocódigo o de diagrama de transición
\end{itemize}
}

%%%%%%%%%%%%%%%%%%%%%%%%%%%%%%%%%%%%%%%%%%%%%%%%%%%%%%%%%%%%%%%%%%%%%%%%%%%%%%%
\subsection{Descripción de las Transacciones}
%%%%%%%%%%%%%%%%%%%%%%%%%%%%%%%%%%%%%%%%%%%%%%%%%%%%%%%%%%%%%%%%%%%%%%%%%%%%%%
\textcolor {red} {Modelo CM-1 que incluya las transacciones identificadas.}


%%%%%%%%%%%%%%%%%%%%%%%%%%%%%%%%%%%%%%%%%%%%%%%%%%%%%%%%%%%%%%%%%%%%%%%%%%%%%%%
\subsection{Especificación de las transacciones}
%%%%%%%%%%%%%%%%%%%%%%%%%%%%%%%%%%%%%%%%%%%%%%%%%%%%%%%%%%%%%%%%%%%%%%%%%%%%%%

\textcolor {red} {
Modelo CM-2 para las transacciones identificadas.
\\
\textbf{NOTA IMPORTANTE:} Normalmente los modelos de comunicación de las prácticas son sencillos, por lo que este formulario puede no ser necesario. Se realizará sólo
para las transacciones complejas que no queden completamente
especificadas con el modelo CM-1.
}

%%%%%%%%%%%%%%%%%%%%%%%%%%%%%%%%%%%%%%%%%%%%%%%%%%%%%%%%%%%%%%%
%FINAL DEL LIBRO
%%%%%%%%%%%%%%%%%%%%%%%%%%%%%%%%%%%%%%%%%%%%%%%%%%%%%%%%%%%%%%%
\end{document}
