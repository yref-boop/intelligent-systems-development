\documentclass[12pt,a4paper,twoside,spanish]{article}      % Libro a 11 pt
\usepackage[utf8]{inputenc}
\usepackage[height=17.5cm,width=13.5cm]{geometry}
\usepackage[spanish]{babel}         % diccionario
\usepackage{epsfig}         % Graficos Postscript
\usepackage{tabularx}
\usepackage{sectsty}
\usepackage{float}
\usepackage{xcolor}



%%%%%%%%%%%%%%%%%%%%%%%%%%%%%%%%%%%%%%%%%%%%%%%
%%%%%%%%%%%%%
%%%%%%%%%%%%% Margenes
%%%%%%%%%%%%%
%%%%%%%%%%%%%%%%%%%%%%%%%%%%%%%%%%%%%%%%%%%%%%%
%%%%% Definimos el maximo tamaño posible.
\marginparwidth 0pt     \marginparsep 0pt
\topmargin   0pt        \textwidth   6.5in
\textheight 23cm

% Margen izq del txt en impares.
\setlength{\oddsidemargin}{.0001\textwidth}

% Margen izq del txt en pares.
\setlength{\evensidemargin}{-.04\textwidth}

% Anchura del texto
\setlength{\textwidth}{.99\textwidth}


%%%%%%%%%%%%%%%%%%%%%%%%%%%%%%%%%%%%%%%%%%%%%%%
%%%%%%%%%%%%%
%%%%%%%%%%%%% Profundidad de enumeracion y tabla de contenidos
%%%%%%%%%%%%%
%%%%%%%%%%%%%%%%%%%%%%%%%%%%%%%%%%%%%%%%%%%%%%%

\setcounter{secnumdepth}{3}
\setcounter{tocdepth}{3}


%%%%%%%%%%%%%%%%%%%%%%%%%%%%%%%%%%%%%%%%%%%%%%%
%%%%%%%%%%%%%
%%%%%%%%%%%%% Nuevos Comandos
%%%%%%%%%%%%%
%%%%%%%%%%%%%%%%%%%%%%%%%%%%%%%%%%%%%%%%%%%%%%%

            %%%%%%%%%%%%%%%%%%%%%%%
            %%%%%%%%%%%%%%%%%%%%%%%
            % Comandos para simplificar
            % la escritura
            %%%%%%%%%%%%%%%%%%%%%%%
            %%%%%%%%%%%%%%%%%%%%%%%

\def\mc{\multicolumn}
            %%%%%%%%%%%%%%%%%%%%%%%
            % Comandos para poder utilizar raggedright en tablas
            %%%%%%%%%%%%%%%%%%%%%%%
\newcommand{\PreserveBackslash}[1]{\let\temp=\\#1\let\\=\temp}
\let\PBS=\PreserveBackslash




%%%%%%%%%%%%%%%%%%%%%%%%%%%%%%%%%%%%%%%%%%%%%%%
%%%%%%%%%%%%%
%%%%%%%%%%%%% Cuerpo del documento
%%%%%%%%%%%%%
%%%%%%%%%%%%%%%%%%%%%%%%%%%%%%%%%%%%%%%%%%%%%%%


\begin{document}

\def\chaptername{Capítulo}
\def\tablename{Tabla}
\def\listtablename{Índice de Tablas}
\chapterfont{\LARGE\raggedleft}

%%%%%%%%%%%%%%%%%%%%%%%%%%%%%%%%%%%%%%%%%%%%%%%%%%%%%%%%%%%%%%%
%%%%%%%%%%%%%%%%%%%%%%%%%%%%%%%%%%%%%%%%%%%%%%%%%%%%%%%%%%%%%%%
% DISEÑO DE LA PAGINA DEL TITULO
%%%%%%%%%%%%%%%%%%%%%%%%%%%%%%%%%%%%%%%%%%%%%%%%%%%%%%%%%%%%%%%
%%%%%%%%%%%%%%%%%%%%%%%%%%%%%%%%%%%%%%%%%%%%%%%%%%%%%%%%%%%%%%%
\pagestyle{empty}

\begin{titlepage}
\setlength{\parindent}{0cm} \setlength{\parskip}{0cm}
\newcommand{\HRule}{\rule{\linewidth}{1mm}}

\vspace*{2cm}
\HRule \\[0.5cm]
\begin{center}
% Letra lineal y negrita
\textsf{\textbf{\large UN SISTEMA BASADO EN CONOCIMIENTO PARA \ldots(tema de la práctica)\\[1.5cm]
Análisis de viabilidad e impacto. \\[0.25cm] Modelado del Contexto en CommonKADS \\[0.5cm]}}
\HRule \vspace*{4cm}

\textsf{\textbf{\normalsize Nombre del miembro 1 del grupo de prácticas\\Miembro 2\\Miembro3\\Miembro 4 \\[1cm]
Grupo de prácticas:\\[3.5cm]}}

\textsf{\textbf{\small Desarrollo de Sistemas Inteligentes\\
Universidade da Coruña \\ Curso 202X}}
\end{center}
\end{titlepage}

\cleardoublepage

%%%%%%%%%%%%%%%%%%%%%%%%%%%%%%%%%%%%%%%%%%%%%%%
%%
%% TABLA DE CONTENIDOS
%%
%%%%%%%%%%%%%%%%%%%%%%%%%%%%%%%%%%%%%%%%%%%%%%%

\pagenumbering{Roman}
\tableofcontents
\cleardoublepage


%%%%%%%%%%%%%%%%%%%%%%%%%%%%%%%%%%%%%%%%%%%%%%%%%%%%%%%%%%%%%%%
%%%%%%%%%%%%%%%%%%%%%%%%%%%%%%%%%%%%%%%%%%%%%%%%%%%%%%%%%%%%%%%
%CONTENIDO DEL DOCUMENTO
%%%%%%%%%%%%%%%%%%%%%%%%%%%%%%%%%%%%%%%%%%%%%%%%%%%%%%%%%%%%%%%
%%%%%%%%%%%%%%%%%%%%%%%%%%%%%%%%%%%%%%%%%%%%%%%%%%%%%%%%%%%%%%%

%numeros arábigos
\pagenumbering{arabic} 
%cabeceras
\pagestyle{myheadings} \markboth{Título reducido} {Modelado Contextual en CommonKADS.}

%indentaciones y espaciado entre párrafos
\setlength{\parindent}{1,5cm} \setlength{\parskip}{0,7cm}

%%%%%%%%%%%%%%%%%%%%%%%%%%%%%%%%%%%%%%%%%%%%%%%%%%%%%%%%%%%%%%%%%%%%%%%%%%%%%%%
%SE PUEDE BORRAR ESTE APARTADO PARA LA ENTREGA
{\color{blue} 

\noindent \textbf{NOTA}: para la entrega, eliminar del documento todas las anotaciones en \textbf{azul}

\paragraph{Indicaciones para la realización y valoración de esta práctica.}

El objetivo del modelo de contexto es determinar la posibilidad y la oportunidad de desarrollar un Sistema Basado en Conocimiento (SBC) tratando de identificar los problemas que podrían hacer peligrar su desarrollo o posterior implantación.

El contenido esperado en esta entrega de prácticas se corresponde con los contenidos de las plantillas ofrecidas por CommonKADS para este primer nivel de análisis.


Es importante entender que CommonKADS es una metodología en espiral y, por tanto, los modelos de contexto se realizan varias veces a lo largo del proyecto para reflejar distintas situaciones. En vuestro caso, \textbf{deberéis realizar los modelos bajo el enfoque inicial}: no hay ningún SBC desarrollado y el modelo de contexto recoge el estado de la organización y el modo de realizar las tareas \textbf{ANTES DEL DESARROLLO DEL SBC}. 

En la valoración de esta práctica se tendrá en cuenta:
\begin{itemize}
    \item La CORRECCIÓN en el empleo de estos modelos
    \item La CALIDAD del trabajo realizado
    \item La CLARIDAD en la redacción de los documentos entregados    
    \item   La participación de TODOS los miembros del grupo  
\end{itemize}
}

\cleardoublepage

%%%%%%%%%%%%%%%%%%%%%%%%%%%%%%%%%%%%%%%%%%%%%%%%%%%%%%%%%%%%%%%%%%%%%%%%%%%%%%%
\section{Análisis de Viabilidad: Modelado de la Organización.}
%%%%%%%%%%%%%%%%%%%%%%%%%%%%%%%%%%%%%%%%%%%%%%%%%%%%%%%%%%%%%%%%%%%%%%%%%%%%%%%



%%%%%%%%%%%%%%%%%%%%%%%%%%%%%%%%%%%%%%%%%%%%%%%%%%%%%%%%%%%%%%%%%%%%%%%%%%%%%%%
\subsection{Formulario OM-1: contexto organizacional, problemas y soluciones.}
%%%%%%%%%%%%%%%%%%%%%%%%%%%%%%%%%%%%%%%%%%%%%%%%%%%%%%%%%%%%%%%%%%%%%%%%%%%%%%%

{\color{blue} 
Identificación de los problemas y oportunidades orientadas al conocimiento de la organización. Si resulta más cómodo, se puede cambiar el formato ``Tabla'' por un desglose en subsecciones utilizando  el comando subsubsection{}
}


\begin{table}[H]
\scriptsize
\begin{tabularx}{\textwidth}{|l|X|} \hline
\textbf{Modelo de Organización} & \textbf{Formulario OM-1: Problemas y Posibilidades de Mejora} \\ \hline\hline

\textsc{Problemas y Oportunidades} & Enumerar los problemas que podemos resolver mediante el SBC y otras posibilidades de mejora que hayan podido observarse.  \\ \hline
\textsc{Contexto Organizacional} & Indicar las características claves del contexto organizacional de la empresa o sector a la que va a ir destinado nuestro producto, así como colocar los
anteriores problemas y posibilidades dentro de la perspectiva adecuada.
Algunas de estas características son (consultad pdf disponible en Moodle sobre este tema):
\begin{enumerate}
    \item Misión, visión y objetivos de la organización 
    \item Factores externos con los que tiene que tratar la organización que pueda afectar a su actividad
    \item Estrategia de la organización
    \item Escala de valores por la que se rige
\end{enumerate}    \\ \hline
\textsc{Soluciones} & Listar las posibles soluciones para cada uno de los problemas y posibilidades percibidas
dentro del contexto de la organización y anotadas al principio de esta tabla. \\
\hline
\end{tabularx}
  \label{tab.OM1}
\end{table}


%%%%%%%%%%%%%%%%%%%%%%%%%%%%%%%%%%%%%%%%%%%%%%%%%%%%%%%%%%%%%%%%%%%%%%%%%%%%%%%
\pagebreak
\subsection{Formulario OM-2: descripción del área de interés de la organización.}
%%%%%%%%%%%%%%%%%%%%%%%%%%%%%%%%%%%%%%%%%%%%%%%%%%%%%%%%%%%%%%%%%%%%%%%%%%%%%%%

{\color{blue}  
Descripción de los aspectos de la organización que tienen impacto y/o se ven afectados
  por las soluciones basadas en conocimiento elegidas. Si resulta más cómodo, se puede cambiar el formato ``Tabla'' por un desglose en subsecciones. }


\begin{table}[H]
\scriptsize
\begin{tabularx}{\textwidth}{|l|X|} \hline
\textbf{Modelo de Organización} & \textbf{Formulario OM-2: Aspectos Variables} \\ \hline\hline

\textsc{Estructura} & Se presenta un gráfico de la parte de la organización bajo análisis en términos de
departamentos, grupos, unidades\ldots\\ \hline
\textsc{Procesos} & Se realiza un diagrama de los procesos que se llevan a cabo y que, posteriormente se descompondrán en tareas en el formulario OM-3. Si solo hubiese un proceso de interés, se indicaría solo un nombre para ese proceso.\\ \hline
\textsc{Personal} &  Se indica qué miembros de la plantilla están implicados en los procesos, como demandantes, destinatarios o proveedores de ese conocimiento. No tienen que ser necesariamente personas físicas, sino que pueden especificarse por el rol
funcional que desarrollan en la organización (director, secretaria, etc.)\\ \hline
\textsc{Recursos} &  Describir los recursos utilizados por los procesos:
\begin{enumerate}
    \item Sistemas de información y otros recursos computacionales.
    \item Equipamiento y material.
    \item Experiencia social o interpersonal que no sea intensiva en conocimiento.
    \item Tecnología, patentes, etc.
\end{enumerate}
\\ \hline
\textsc{Conocimiento} &  Representa un recurso especial explotado en el proceso. Debido a la
importancia de este aspecto este componente dispone de un formulario aparte (OM-4). Aquí solo es necesario indicar ``véase OM-4'' \\ \hline
\textsc{Cultura y Potencial} &  En este apartado se trata de reflejar aquellos \textit{modus operandi} que no están explícitos en la descripción formal de la organización, incluyendo formas de trabajar, de comunicarse y relaciones formales e informales.\\ \hline
\end{tabularx}
  \label{tab.OM2}
\end{table}



%%%%%%%%%%%%%%%%%%%%%%%%%%%%%%%%%%%%%%%%%%%%%%%%%%%%%%%%%%%%%%%%%%%%%%%%%%%%%%%
\pagebreak
\subsection{Formulario OM-3: descomposición del proceso de negocio.}
%%%%%%%%%%%%%%%%%%%%%%%%%%%%%%%%%%%%%%%%%%%%%%%%%%%%%%%%%%%%%%%%%%%%%%%%%%%%%%%

{\color{blue} Descripción del proceso de interés a partir de las tareas que lo componen. En este apartado es muy importante identificar y distinguir los recursos de conocimiento formales y los no formales. }

\begin{table}[H]
\scriptsize
\begin{tabularx}{\textwidth}{|p{0.2cm}|>{\raggedright}X|>{\raggedright}X|>{\raggedright}X|>{\raggedright}X|>{\raggedright}X|>{\PBS\raggedright}X|} \hline
\multicolumn{3}{|l}{\textbf{Modelo de Organización}} &
\multicolumn{4}{|l|}{\textbf{Formulario OM-3: Descomposición de
los Procesos}}\\ \hline\hline \textsc{N\textordmasculine} &
\textsc{Tarea} &  \textsc{Realiza\-da por} &  \textsc{¿Dónde?} &
\textsc{Recursos de Conocimiento} & \textsc{¿In\-ten\-si\-va en
Conocimiento?} & \textsc{Im\-por\-tan\-cia} \\
\hline Id. & Nombre \emph{(partes de los procesos descritos en OM-2)}& Agente
humano/ soft\-wa\-re \emph{(per\-so\-nal/re\-cur\-sos de OM-2)} &
Localización en la estructura \emph{(\textit{véase} OM-2)} & Nombre \emph{(Se
detalla en OM-4)} & (sí o no) & \emph{(usando un criterio definido
por el Ingeniero de Conocimiento, por ejemplo escala 1--5)} \\ \hline
\end{tabularx}
\label{tab.OM3}
\end{table}



%%%%%%%%%%%%%%%%%%%%%%%%%%%%%%%%%%%%%%%%%%%%%%%%%%%%%%%%%%%%%%%%%%%%%%%%%%%%%%%
\pagebreak
\subsection{Formulario OM-4: activos de conocimiento.}
%%%%%%%%%%%%%%%%%%%%%%%%%%%%%%%%%%%%%%%%%%%%%%%%%%%%%%%%%%%%%%%%%%%%%%%%%%%%%%%

{\color{blue} Descripción del componente \textit{conocimiento} implicado en el proceso y tareas bajo estudio. Es importante añadir comentarios en las columnas donde se indica, cuando la respuesta sea no.
}

\begin{table}[H]
\scriptsize
\begin{tabularx}{\textwidth}{|p{1.3cm}|p{1.3cm}|p{1.3cm}|X|X|X|X|} \hline
\multicolumn{3}{|l}{\textbf{Modelo de Organización}} & \multicolumn{4}{|l|}{\textbf{Formulario OM-4: Activos de Conocimiento}} \\ \hline\hline
\textsc{Recurso de Conocimiento} & \textsc{Pertenece a} &  \textsc{Usado en} &  \textsc{¿Forma
Correcta?} & \textsc{¿Lugar Correcto?} & \textsc{¿Tiempo Correcto?} & \textsc{¿Calidad Correcta?}\\ \hline
Nombre \emph{(\textit{véase} OM-3)} & Agente \emph{(\textit{véase} OM-3)}& Tarea \emph{(\textit{véase} OM-3)}& (si o no; comentario)(*1)&
(si o no; comentario)& (si o no; comentario) & (si o no; comentario) (*2)\\ \hline
\end{tabularx}

  \label{tab.OM4}
\end{table}

\noindent (*1) Ejemplo de cómo incluir un comentario cuando la respuesta sea no

\noindent (*2) Ejemplo de cómo incluir un comentario cuando la respuesta sea no

%%%%%%%%%%%%%%%%%%%%%%%%%%%%%%%%%%%%%%%%%%%%%%%%%%%%%%%%%%%%%%%%%%%%%%%%%%%%%%%
\pagebreak
\subsection{Formulario OM-5: Análisis de viabilidad.}
%%%%%%%%%%%%%%%%%%%%%%%%%%%%%%%%%%%%%%%%%%%%%%%%%%%%%%%%%%%%%%%%%%%%%%%%%%%%%%%

{\color{blue}
Este apartado del documento \textbf{no debe redactarse en formato pregunta-respuesta} sino que debe ser una narración que de respuesta a las preguntas que sean relevantes en nuestro proyecto. Si resulta más cómodo, se puede cambiar el formato ``Tabla'' por un desglose en subsecciones. 

 En la viabilidad empresarial del proyecto, una de las cuestiones que tenemos que resolver es el coste de acometer el proyecto. Para el análisis de costes es necesario tener en cuenta las etapas del proyecto, la duración estimada de cada una y los perfiles que intervienen en el proyecto. Aunque CommonKads hace una gestión de proyecto en espiral, para el análisis de costes, supondremos las etapas en cascada. Estas son:
 \begin{itemize}
 	\item Análisis del contexto
 	\item Modelado conceptual (implica la extracción de conocimiento y formalización de los algoritmos que implementará el sistema)
 	\item Diseño del software
 	\item Implementación
 	\item Pruebas 	
 \end{itemize}

Los perfiles que estarán implicados (aunque no todos en todas las etapas) son: 
\begin{itemize}
	\item Ingeniero/a de conocimiento: responsable de los análisis, extracción y verificación del conocimiento
	\item Programador/a senior
	\item Programador/a junior
\end{itemize}

Estimando la duración de las etapas y teniendo en cuenta el esfuerzo y el coste/hora de cada perfil en cada etapa (hay guías salariales disponibles en internet) tendremos parte del coste del proyecto.

Una última cuestión a tener en cuenta serán los recursos necesarios para el desarrollo que puede ir desde sistemas que haya que comprar hasta expertos a los que haya que pagar o datos o testers necesarios para las pruebas.  

}

\begin{table}[H]
\scriptsize
\begin{tabularx}{\textwidth}{|l|X|} \hline


\textbf{Modelo de Organización} & \textbf{Formulario OM-5: Elementos del Documento de Viabilidad}\\ \hline\hline
\textsc{Viabilidad Empresarial} & Se deben responder las siguientes preguntas para cada problema y posibilidad e incluir la solución propuesta:
\begin{enumerate}
    \item¿Cuáles son los beneficios esperados para la organización con la adopción de la
solución? Se deben identificar los beneficios tanto económicos como intangibles.
    \item¿Cuánto es el valor añadido esperado?
    \item¿Cuál es el coste esperado de la solución? 
    \item¿Y comparado con otras soluciones alternativas?
    \item¿Se requieren cambios organizacionales?
    \item¿Hasta qué punto están implicados los riesgos e incertidumbres de la empresa
en la solución propuesta?
\end{enumerate}
\\ \hline
\end{tabularx}
\caption{Formulario OM-5 (parte I).}
  \label{tab.OM5_1}
\end{table}

\begin{table}[H]
\scriptsize
\begin{tabularx}{\textwidth}{|l|X|} \hline

\textbf{Modelo de Organización} & \textbf{Formulario OM-5: Elementos del Documento de Viabilidad}\\ \hline\hline

\textsc{Viabilidad Técnica} & Se deben responder las siguientes preguntas para cada problema y posibilidad junto
a la solución propuesta:
\begin{enumerate}
    \item ¿Qué complejidad tiene la tarea que desarrollará el SBC desde el punto de vista del
conocimiento a almacenar y los procesos de razonamiento a realizar?
¿Son las técnicas y métodos actuales adecuados?
    \item ¿Existen aspectos críticos relativos al tiempo, calidad, recursos necesarios, etc.?
¿Qué se va a hacer al respecto?
    \item ¿Están claros los criterios bajo los cuales vamos a determinar el éxito del
proyecto? ¿Cómo vamos a comprobar la calidad y la validez del sistema?
    \item ¿Cómo son de complejas las interfaces con el usuario que se necesitan? ¿Son las técnicas y métodos actuales adecuados?
    \item ¿Cómo es de compleja la interacción con otros sistemas de información y otros
recursos potenciales (integración de sistemas, interoperatividad)? ¿Son las técnicas y métodos actuales adecuados?
    \item ¿Existen riesgos e incertidumbres tecnológicos adicionales?
\end{enumerate}   \\ \hline


\textsc{Viabilidad del Proyecto} & Se deben responder las siguientes preguntas para cada problema y posibilidad junto
a la solución propuesta:
\begin{enumerate}
    \item ¿Existe el compromiso adecuado por parte del personal (gestores, expertos,
usuarios, clientes, equipo del proyecto) para llevar a cabo los siguientes pasos
del proyecto?
    \item ¿Están disponibles los recursos necesarios (en tiempo, presupuesto,
equipamiento, personal)?
    \item ¿Están disponibles el conocimiento y las capacidades requeridas?
    \item ¿Son realistas las expectativas sobre el proyecto y sus resultados?
    \item ¿Es adecuada la organización del proyecto y las comunicaciones externas o
internas?
    \item ¿Existen riesgos e incertidumbres adicionales relativos al proyecto?
\end{enumerate}
\\ \hline
\textsc{Acciones Propuestas} & Esta parte del documento de viabilidad está directamente relacionada con los acuerdos de gestión y la toma de decisiones. Valora e
integra en un plan de
actuación concreto los resultados obtenidos en el análisis previo. Se va completando a lo largo de todo el ciclo de vida del proyecto (en espiral). Para esta primera etapa, tan solo es necesario responder a las preguntas 1 y 2.  
\begin{enumerate}
    \item ¿Qué área de actuación se recomienda entre las áreas de problemas/oportunidades encontradas y descritas en OM-1?
    \item ¿Cuál es la solución recomendada para el área elegida y qué tareas de OM-3 deberían automatizarse?
    \item ¿Cuáles son los resultados, costes y beneficios esperados?
    \item ¿Qué acciones deben de tomar dentro del proyecto para alcanzar dicha
solución?
    \item Si las circunstancias que rodean a la organización cambian (tanto externas como
internas), ¿en qué condiciones es apropiado reconsiderar las decisiones
tomadas?
\end{enumerate}
\\ \hline
\end{tabularx}
\caption{Formulario OM-5 (parte II).}
  \label{tab.OM5_2}
\end{table}


%%%%%%%%%%%%%%%%%%%%%%%%%%%%%%%%%%%%%%%%%%%%%%%%%%%%%%%%%%%%%%%%%%%%%%%%%%%%%%%
\pagebreak
\section{Análisis de Impactos y Mejoras: Modelado de las Tareas y los Agentes.}
%%%%%%%%%%%%%%%%%%%%%%%%%%%%%%%%%%%%%%%%%%%%%%%%%%%%%%%%%%%%%%%%%%%%%%%%%%%%%%%

{\color{blue} Los modelos TM-1 y TM-2 solo los tendréis que realizar para las tareas que identifiquéis como \textit{intensivas en conocimiento} en el OM-3 y que vayan a ser implementadas en el sistema final (es decir, las elegidas al final de OM-5).}

%%%%%%%%%%%%%%%%%%%%%%%%%%%%%%%%%%%%%%%%%%%%%%%%%%%%%%%%%%%%%%%%%%%%%%%%%%%%%%%
\subsection{Formulario TM-1: análisis de tareas.}
%%%%%%%%%%%%%%%%%%%%%%%%%%%%%%%%%%%%%%%%%%%%%%%%%%%%%%%%%%%%%%%%%%%%%%%%%%%%%%%

{\color{blue}Descripción detallada de tareas en el contexto del proceso de interés.}

\begin{table}[H]
\scriptsize
\begin{tabularx}{\textwidth}{|l|X|} \hline

\textbf{Modelo de Tareas} & \textbf{Formulario TM-1: Análisis de Tareas} \\ \hline\hline
\textsc{Tarea} & Identificador (N\textordmasculine) y nombre de tarea \emph{(Ref. OM-3)}\\ \hline
\textsc{Organización}  & Proceso del que esta tarea forma parte, y
parte de la organización donde se desarrolla \emph{(Ref. OM-3)}.\\ \hline
\textsc{Objetivo y valor} &  Objetivo de la tarea y el valor que añade al proceso
del que forma parte cuando es ejecutada.\\ \hline
\textsc{Dependencia y Flujos} & \textit{1. Tareas precedentes:} Tareas que se ejecutan antes que esta tarea y le
proporcionan las entradas.\\
 &  \textit{2. Tareas que le siguen:} Tareas que se ejecutan después de esta tarea y que utilizan
algunas de sus salidas como entradas. \\ \hline
\textsc{Objetos manipulados} & 1. Objetos de entrada de la tarea.\\
  &  2. Objetos de salida de la tarea.\\
  &  3. Objetos internos: Objetos importantes (si los hay) internos a la tarea y que no
son parte de la entrada y de la salida. \\
& \emph{Todos estos objetos incluyen elementos de información y conocimiento.}
\\ \hline
\textsc{Tiempo y control} & \textit{1. Frecuencia y duración:} cada cuánto tiempo se ejecuta la tarea y cuánto tiempo
consume normalmente.\\
& \textit{2. Control:} Relación de control con respecto a otras tareas.\\
& \textit{3. Restricciones:} Precondiciones y poscondiciones de la tarea,
y restricciones que deben ser satisfechas durante su ejecución.
\\ \hline
\textsc{Agentes} &
El personal y/o los sistemas de información que son
responsables de llevar a cabo la tarea \emph{(Referencia a OM-2:
Personal, recursos; OM-3: ejecutada por)}.\\ \hline
\textsc{Conocimiento y
Capacidad} &
Capacidades necesarias para el desarrollo de la tarea \emph{(Referencia a OM-4)}. Para
especificar los elementos de conocimiento implicados se utiliza un formulario adicional (TM-2).
También se deben indicar otras habilidades y capacidades
requeridas e indicar qué parte de la tarea es intensiva en
conocimiento. Finalmente, se deben indicar las capacidades
que las tareas proporcionan a la organización. \\ \hline
\textsc{Recursos} &
Se describen \emph{(Refinamiento de OM-2)} y, si se puede, se cuantifican los recursos
requeridos por la tarea (personal, sistemas y equipamiento,
materiales, partidas presupuestarias, etc.). \\ \hline
\textsc{Calidad y eficiencia} &
Enumerar las medidas de calidad y eficiencia que utiliza la
organización para determinar la ejecución exitosa de la
tarea.
\\ \hline
\end{tabularx}

  \label{tab.TM1}
\end{table}



%%%%%%%%%%%%%%%%%%%%%%%%%%%%%%%%%%%%%%%%%%%%%%%%%%%%%%%%%%%%%%%%%%%%%%%%%%%%%%%
\pagebreak
\subsection{Formulario TM-2: análisis de los cuellos de botella del conocimiento.}
%%%%%%%%%%%%%%%%%%%%%%%%%%%%%%%%%%%%%%%%%%%%%%%%%%%%%%%%%%%%%%%%%%%%%%%%%%%%%%%

{\color{blue} Especificación del conocimiento que se emplea en una tarea, sus cuellos de botella y posibles mejoras. 

La última columna de la tabla son dos preguntas que deben responderse por separado. Los cuellos de botella se refieren a si la naturaleza de ese elemento de conocimiento supone un problema a la hora de realizar la tarea correctamente. La segunda pregunta ``¿Debe ser mejorado?'' determina si ese cuello de botella tiene o puede superarse con la implementación del sistema inteligente o, por el contrario, no es posible hacerlo. Además, en esta columna no basta con una respuesta sí/no, sino que debe incluir una breve explicación sobre la respuesta. }




\begin{table}[H]
\scriptsize
\begin{tabularx}{\textwidth}{|p{5cm}|>{\PBS\raggedright}p{0.8cm}|X|} \hline
\textbf{Modelo de Tareas} & \multicolumn{2}{l|}{\textbf{Formulario TM-2: Elemento de Co\-no\-ci\-mien\-to}} \\ \hline\hline
\textsc{Nombre} &  \multicolumn{2}{l|}{Elemento de conocimiento \emph{(Referencia a OM-3)}}\\ \hline
\textsc{Poseído por} &  \multicolumn{2}{X|}{Agente \emph{(Referencia a OM-4)}}\\ \hline
\textsc{Usado en} &  \multicolumn{2}{l|}{Nombre de la tarea e identificador \emph{(Referencia a OM-3)}}\\ \hline
\textsc{Dominio} &  \multicolumn{2}{p{7.5cm}|}{Dominio más amplio en el que se encuentra el conocimiento (campo de la especialidad,
disciplina, rama de la ciencia o ingeniería, etc.).}\\ \hline

\textbf{Naturaleza del conocimiento} & \emph{(Sí/No)} &
\textbf{¿Supone un cuello de botella?¿Debe ser mejorado?}\\ \hline Formal,
riguroso & & \\ \hline Empírico, cuantitativo& & \\ \hline
Heurístico, sentido común& & \\ \hline Altamente especializado,
específico del dominio& & \\ \hline Basado en la experiencia& & \\
\hline Basado en la acción$^1$& & \\ \hline Incompleto& & \\
\hline Incierto, puede ser incorrecto & & \\ \hline Cambia con
rapidez& &
\\ \hline Difícil de verificar& & \\ \hline Tácito, difícil de
transferir& & \\ \hline \textbf{Forma del conocimiento} & &\\ \hline
Mental& & \\ \hline Papel& & \\ \hline Electrónica& & \\ \hline
Habilidades& & \\ \hline Otros& & \\ \hline \textbf{Disponibilidad
del Conocimiento} &  &\\ \hline Limitaciones en tiempo& & \\ \hline
Limitaciones en espacio& & \\ \hline Limitaciones de acceso& & \\
\hline Limitaciones de calidad& & \\ \hline Limitaciones de forma& &
\\ \hline
\end{tabularx}
  \label{tab.TM2}
\end{table}

{\scriptsize \noindent $^1$ Se refiere a conocimiento que se
adquiere con la repetición de actividades físicas tales como
conducir, encestar, etc.}

%%%%%%%%%%%%%%%%%%%%%%%%%%%%%%%%%%%%%%%%%%%%%%%%%%%%%%%%%%%%%%%%%%%%%%%%%%%%%%%
\pagebreak
\subsection{Formulario AM-1: descripción de los agentes.}
%%%%%%%%%%%%%%%%%%%%%%%%%%%%%%%%%%%%%%%%%%%%%%%%%%%%%%%%%%%%%%%%%%%%%%%%%%%%%%%

{\color{blue} Descripción de los agentes implicados en las tareas de interés.}

\begin{table}[H]
\scriptsize
\begin{tabularx}{\textwidth}{|l|X|} \hline
\textbf{Modelo de Agentes} & \textbf{Formulario AM-1: Agentes} \\ \hline\hline
\textsc{Nombre} &  Nombre del agente \\ \hline
\textsc{Organización} &  Indica la posición del agente dentro de la organización, incluyendo su tipo (humano, sistema de información), etc. \emph{(referencia a OM-2)}\\ \hline
\textsc{Implicado en} & Tareas en las que está implicado \emph{(referencia a TM-1)} \\ \hline
\textsc{Se comunica con} &  Lista de nombres de otros agentes  \\ \hline
\textsc{Conocimiento} &  Elementos de conocimiento que el agente posee \emph{(referencia a TM-2)} \\ \hline
\textsc{Otras Competencias} &  Lista del resto de las competencias requeridas o presentes en el agente \\ \hline
\textsc{Responsabilidades y restricciones} &  Responsabilidades que el agente tiene durante la ejecución de las
tareas, así como de sus restricciones en autoridad, respecto a normas legales o
profesionales, etc.\\ \hline
\end{tabularx}
 \label{tab.AM1}
\end{table}


%%%%%%%%%%%%%%%%%%%%%%%%%%%%%%%%%%%%%%%%%%%%%%%%%%%%%%%%%%%%%%%%%%%%%%%%%%%%%%%
\pagebreak
\section{Conclusiones del análisis de viabilidad e impacto.}
%%%%%%%%%%%%%%%%%%%%%%%%%%%%%%%%%%%%%%%%%%%%%%%%%%%%%%%%%%%%%%%%%%%%%%%%%%%%%%%


%%%%%%%%%%%%%%%%%%%%%%%%%%%%%%%%%%%%%%%%%%%%%%%%%%%%%%%%%%%%%%%%%%%%%%%%%%%%%%%
\subsection{Formulario OTA-1: recomendaciones y acciones.}
%%%%%%%%%%%%%%%%%%%%%%%%%%%%%%%%%%%%%%%%%%%%%%%%%%%%%%%%%%%%%%%%%%%%%%%%%%%%%%%



{\color{blue} Lista de control para el documento de toma de decisiones sobre impactos y mejoras.

\textbf{NOTA IMPORTANTE: } Este documento solo es necesario cuando se identifican varios problemas y se proponen varias soluciones en OM-1 y, en consecuencia, es necesario hacer varios OM-5. Por tanto, \textbf{no es necesario realizarlo para las prácticas.} }




\vspace*{1cm}


\begin{table}[H]
\scriptsize
\begin{tabularx}{\textwidth}{|p{4cm}|X|} \hline
\textbf{Modelos de Organización, Tareas y Agentes} &
\textbf{Formulario OTA-1: Documento sobre Impactos y Mejoras}\\
\hline\hline \textsc{Impactos y Cambios en la Organización} &
Describir los impactos y los cambios que el SBC traerá a la
organización, comparándola con la estructura organizativa actual
\emph{(descrita en OM-2)}, y con cómo será dicha organización en
el futuro. Esta descripción se realizará de forma global para los
seis aspectos variables descritos en OM-2.
 \\ \hline
\textsc{Impactos y Cambios en Tareas y Agentes} & Describir los
impactos y los cambios que el SBC introducirá en los agentes y las
tareas, comparándolos con la situación actual \emph{(descrita en
TM-1/2 y AM-1)}. Es importante que no sólo se considere al
personal implicado en una determinada tarea, sino también a otros
agentes y poseedores de conocimiento (gestores, usuarios,
clientes): \\
& 1. Cambios en la estructura de las tareas (flujo, dependencias,
objetos manipulados, temporización, control). \\
& 2. Cambios en los recursos necesarios. \\
& 3. Criterios para evaluar la calidad y el rendimiento.\\
& 4. Cambios en la plantilla y en los agentes implicados.\\
& 5. Cambios en las posiciones, responsabilidades, autoridad y limitaciones de los
individuos implicados en la ejecución de la tarea.\\
& 6. Cambios en el conocimiento y capacidad requeridos.\\
& 7. Cambios en los canales de comunicación.\\
\hline
\textsc{Actitudes y Compromisos} &  Analizar cómo reaccionarán a
los cambios introducidos los individuos y el personal cualificado
involucrado, y decidir si existe la base suficiente para llevar a
cabo dichos cambios.
 \\ \hline
\textsc{Acciones Propuestas} &  Apartado directamente relacionado
con los acuerdos de gestión y la toma de decisiones. Los
resultados obtenidos con el análisis previo se valoran e integran
en un plan de actuación concreto: \\
& \textit{1. Mejoras:} ¿Cuáles son los cambios recomendados (organización, tareas, personal, sistemas\ldots)?\\
& \textit{2. Medidas adicionales:} ¿Qué medidas hay que tomar para facilitar dichos cambios? (formación,
facilidades\ldots)\\
& \textit{3. Acciones del proyecto:} ¿Cuál es la siguiente acción a realizar dentro del proyecto respecto a la
solución basada en el conocimiento asumida?\\
& \textit{4. Resultados, costes y beneficios esperados} \emph{(referencia a OM-5)}.\\
& 5. Si las circunstancias que rodean a la organización cambian (tanto externas como
internas), ¿en qué condiciones es apropiado reconsiderar las
decisiones tomadas? \\ \hline

\end{tabularx}
  \label{tab.OTA1}
\end{table}


%%%%%%%%%%%%%%%%%%%%%%%%%%%%%%%%%%%%%%%%%%%%%%%%%%%%%%%%%%%%%%%
%FINAL DEL LIBRO
%%%%%%%%%%%%%%%%%%%%%%%%%%%%%%%%%%%%%%%%%%%%%%%%%%%%%%%%%%%%%%%
\end{document}
